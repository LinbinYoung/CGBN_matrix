\documentclass{article}
\usepackage{ctex}
\usepackage{graphicx}
\usepackage{listings}
\usepackage{caption2}
\usepackage{subfigure}
\usepackage{url}
\usepackage{float}
\usepackage{xcolor}
\usepackage{color}
\usepackage{ulem}
\lstset{
    columns=fixed,       
    numbers=left,                                        % 在左侧显示行号
    frame=none,                                          % 不显示背景边框
    backgroundcolor=\color[RGB]{245,245,244},            % 设定背景颜色
    keywordstyle=\color[RGB]{40,40,255},                 % 设定关键字颜色
    numberstyle=\footnotesize\color{darkgray},           % 设定行号格式
    commentstyle=\it\color[RGB]{0,96,96},                % 设置代码注释的格式
    stringstyle=\rmfamily\slshape\color[RGB]{128,0,0},   % 设置字符串格式
    showstringspaces=false,                              % 不显示字符串中的空格
    language=c++,                                        % 设置语言
}
\title{CUDA api}
\author{杨林彬@Clustar.ai}   %作者的名称
\date{\today}       %日期
\pagebreak 
% 设置页面的环境,a4纸张大小,左右上下边距信息
\usepackage[a4paper,left=10mm,right=10mm,top=15mm,bottom=15mm]{geometry} 
\begin{document}
\maketitle
%生成目录设置
\renewcommand{\contentsname}{Contents}
\tableofcontents
\pagebreak
% \centerline{{《江湖》}}
% \centerline{{飞雪连天射白鹿,}}
% \centerline{{笑书神侠倚碧鸳。}}
% \centerline{{红尘有你终相伴,}}
% \centerline{{笑傲江湖情两牵。}}
% \pagebreak
% %摘要开始部分
% \begin{abstract}
% 虽然极大可能是回不了UC San Diego了,但是热爱生活心不会改变。 \par
% 这篇报告主体内容是本人自2020年1月6日入职以来的学习历程。主要围绕着CUDA学习,高精度数值(Multiple Precision Number)运算,联邦学习(Federated Learning)以及对同态加密Pailler算法的的一点认识。\par
% 静下心来,智慧升起。
% \end{abstract}
% \pagebreak
%标题开始
\end{document}